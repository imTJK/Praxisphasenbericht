
\section {Notizen zur Arbeit}

Hier kommen die Notizen zur Arbeit an den jeweiligen Tagen hinein.

Angefangen hat das ganze am 07.07.2025.

\paragraph{07.07} 
TMC5160-Treiber (weiter-) geschrieben.
\begin{description}
    \item Rampenprofil-Kalkulationen vervollständigt (Grafik einbinden)
    \item Error-Handling ermöglicht aber nicht integriert
    \item Unterschiede zwischen skalierten uint64\_t und standard float-Präzision untersucht
    \begin{itemize}
        \item Vergleich sf\_stepper.cpp vs sf\_stepper\_u64.cpp
        \item Grafik einbinden
        \item Messungen bei verschiedenen Geschwindigkeiten
    \end{itemize}
    \item C-Wrapper Funktionen angepasst
\end{description}


\paragraph{09.07}
TMC5160-Treiber "vervollständigt".
\begin{description}
    \item IRQ-Callback für das Modul eingefügt
    \SubItem {SF\_Stepper\_ModulIRQ\_Callback}
    \item SPI-Task Interrupt Callback eingefügt
    \SubItem {SF\_Stepper\_DI\_Int\_Callback}
    \item ISR-sichere Queue funktionen in den SPI-Task eingefügt
    \SubItem {SPI\_Task\_Send\_Request\_fromISR \& SPI\_Task\_QueueRequest\_fromISR}
    \item Fernseher- und "Hologram"-Fan-Stand für Messen aufgebaut, überprüft und repariert.
    \SubItem {Bild einfügen}
    \item Fernseher in Neotel-Büro aufgebaut
\end{description}

\paragraph{22.07 - 29.07}
Vorige Woche komplett krank
\begin{description}
    \item Command Interpreter für den W5500 integriert\\
    Simples unifizieren, parsen und auswerten
    \SubItem{W5500::parseAndExecuteCommand}
    \SubItem{W5500::processIncomingMessage}
    \SubItem{W5500::sendResponse}
    \SubItem{W5500::handleCommand}
    \item TMC5160 Stepper-Driver debugging
    \SubItem{calculateRampParameters updated}
    \SubItem{Calculation spreadsheet updated}
    \SubItem{handleEvent updated}
\end{description}

\paragraph{03.08 - 08.08}
\begin{description}
    \item SF-Stepper Modul "beendet"
    \item Raspberry Pi Pico + TinyUSB \& Pico Pio USB basierten Maussensor-Parser geschrieben zum testen vom LV
    \SubItem {Simple DPI-Kalibrierung}
    \SubItem {Session-basiertes Messverfahren}
    \SubItem {Bild einfügen, mögliches Schaltbild vom Pico}
    \item Halterung + LV-Integration für die Maus konstruiert und 3D-gedruckt
    \SubItem Bild einfügen (Konstruktion + weitere Designs)
\end{description}

\paragraph{08.08-24.11}
Hauptprojekt (Ethernet-Driver Integration und Compatibility-Layer) beendet
\begin{description}
    \item Wechsel von .cpp zu nativer .c Implementation
    \item Vollständige integration in vorhandene Firmware
    \SubItem {Registrierung der Pins des Addon-Boards}
    \SubItem {Registrierung des Drivers in dem Bus-Device Handler}
    \SubItem {Schreiben von Compatibility-Layer für CI und weitere Module}
    \item Entwicklung weiterer funktionalität (nach Plan) im iterativen Entwicklungsprozess
    \SubItem {Statisch allokierte Read-/Write-Buffer Pools um dynamische Partitionierung während der Runtime ohne Speichermanipulation zu ermöglichen}
    \SubItem {Status, ethernet\_interface und socket Handle-Strukturen}
    \SubItem {SPI-reliant funktionen in jeweiligen SPI-Task übergeben}
    \SubItem {async read-write-funktionen für alle SPI-Devices geschriben mit kontext-basierten freeRTOS-Calls / Context-Switching}
    
\end{description}