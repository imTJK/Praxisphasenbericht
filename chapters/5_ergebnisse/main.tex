\chapter{Ergebnisse und Evaluation}
\label{chap:ergebnisse}

Dieses Kapitel präsentiert die erzielten Ergebnisse der drei Hauptprojekte und evaluiert deren Qualität anhand definierter Kriterien. Die Bewertung erfolgt sowohl quantitativ als auch qualitativ.

\section{TMC5160 Stepper-Motor-Driver}
\label{sec:ergebnisse_tmc}

\subsection{Erreichte Ziele}

Die Implementierung der Rampenprofil-Berechnungen wurde erfolgreich abgeschlossen:

\begin{itemize}
    \item \textbf{Funktionale Vollständigkeit:} Alle spezifizierten Berechnungen wurden implementiert
    \item \textbf{Validierung:} Eingabevalidierung und Konsistenzprüfung funktionieren zuverlässig
    \item \textbf{Integration:} Nahtlose Einbindung in die ATNC3-Firmware
    \item \textbf{Genauigkeit:} Zeitliche Abweichung <3ms, räumliche Abweichung <1 Fullstep
\end{itemize}

\subsection{Qualitative Bewertung}

Die mathematische Modellierung erwies sich als robust und präzise. Die iterative Entwicklung mit regelmäßigen Code-Reviews führte zu einer wartbaren Code-Struktur. Die defensive Programmierung mit expliziter Fehlerbehandlung erhöht die Zuverlässigkeit im Produktiveinsatz.

\section{Hardware-Testing-System}
\label{sec:ergebnisse_hardware}

\subsection{Erreichte Ziele}

Das Projekt zur Verifikation der TMC5160-Rampenprofile lieferte gemischte Ergebnisse:

\nlparagraph{Erfolgreiche Aspekte}
\begin{itemize}
    \item USB-Host-Implementierung auf dem RP2040 funktioniert zuverlässig
    \item TinyUSB-Integration und PIO USB wurden erfolgreich umgesetzt
    \item Mechanische Integration mittels 3D-Druck war praktikabel
    \item Projektdauer von 6 Tagen wurde eingehalten
\end{itemize}

\nlparagraph{Limitierungen}
\begin{itemize}
    \item Messgenauigkeit für die Verifikation der Rampenprofile nicht ausreichend (>5\% Fehler)
    \item Sporadischer Tracking-Verlust bei inhomogenen Oberflächen
    \item Fehlende Reproduzierbarkeit zwischen Sessions
    \item Testergebnisse für TMC5160-Validierung unzuverlässig
\end{itemize}

\subsection{Bewertung}

Der Prototyp demonstrierte die technische Machbarkeit des USB-Host-Konzepts, erwies sich jedoch als ungeeignet für die Verifikation der Lötvorschub-Bewegungen. Die ungenauen Messergebnisse verhinderten eine belastbare Aussage über die Korrektheit der TMC5160-Rampenprofil-Implementierung.

Das Projekt erfüllte dennoch seinen Zweck als Rapid-Prototyping-Ansatz, da die technischen Grenzen des gewählten Sensorprinzips frühzeitig identifiziert wurden und alternative Messverfahren für zukünftige Tests evaluiert werden konnten.

\section{W5500 Ethernet-Driver Integration}
\label{sec:ergebnisse_w5500}

\subsection{Erreichte Ziele}

Das Hauptprojekt wurde erfolgreich mit folgenden Ergebnissen abgeschlossen:

\nlparagraph{Funktionale Ergebnisse}
\begin{itemize}
    \item Vollständig funktionsfähiger W5500 Ethernet-Driver in nativer C-Implementierung
    \item Nahtlose Integration in die ATNC3-Firmware-Architektur
    \item Ethernet-Interface ermöglicht anderen Modulen die Nutzung der Netzwerkfunktionalität
    \item Statisches Buffer-Management-System für deterministische Speicherverwaltung
    \item Asynchrone SPI-Kommunikation mit freeRTOS-Context-Switching
\end{itemize}

\nlparagraph{Qualitätsmerkmale}
\begin{itemize}
    \item Produktionsreife Code-Qualität nach mehreren Code-Review-Zyklen
    \item Dokumentierte Schnittstellen für zukünftige Erweiterungen
    \item Kompatibilität mit bestehenden und zukünftigen Modulen
\end{itemize}

\subsection{Quantitative Bewertung}

% TODO: Weitere messbare Metriken ergänzen? (z.B. Lines of Code, Durchsatz, Latenz)

\begin{table}[ht]
    \centering
    \begin{tabular}{l|c|c}
        Metrik & Ziel & Erreicht \\
        \hline
        Projektdauer & 14 Wochen & 15 Wochen \\
        Code-Reviews bestanden & -- & 4 Zyklen \\
        Buffer-Allokation & O(1) & O(1) \\
        Speicherfragmentierung & 0\% & 0\% \\
        Integration in Firmware & vollständig & vollständig \\
    \end{tabular}
    \caption{Quantitative Projektbewertung W5500}
    \label{tab:w5500_metrics}
\end{table}

\subsection{Qualitative Bewertung}

\nlparagraph{Innovative Aspekte}
Das statische Buffer-Pool-System stellt eine elegante Lösung für ein klassisches Embedded-Problem dar. Die Kombination aus Compile-Zeit-Allokation und Runtime-Verwaltung ermöglicht sowohl Flexibilität als auch Determinismus.

Die asynchrone SPI-Architektur demonstriert modernes Embedded-Design: Statt blockierender Calls werden die RTOS-Mechanismen optimal genutzt, was zu besserer Responsiveness führt.

\nlparagraph{Herausforderungen}
Die größten Herausforderungen waren:
\begin{itemize}
    \item Race-Conditions in Multi-Task-Umgebung
    \item Integration in Legacy-Code bei gleichzeitiger Wahrung der Kompatibilität
    \item Debugging asynchroner Systeme
\end{itemize}

\section{Gesamtbewertung der Praxisphase}
\label{sec:gesamtbewertung}

\subsection{Erfüllung der Zielsetzung}

Die in Kapitel~\ref{sec:zielsetzung} definierten Ziele wurden wie folgt erfüllt:

\begin{enumerate}
    \item \textbf{TMC5160 Stepper-Motor-Driver:} Vollständig erreicht. Die Rampenprofil-Kalkulationen sind produktionsreif implementiert.
    
    \item \textbf{Hardware-Testing:} Teilweise erreicht. Die technische Implementierung war erfolgreich, die Anwendung für den spezifischen Use-Case jedoch nicht geeignet.
    
    \item \textbf{W5500 Ethernet-Integration:} Vollständig erreicht. Der Driver ist produktionsreif in die Firmware integriert.
\end{enumerate}

\subsection{Kompetenzentwicklung}

\nlparagraph{Neu erworbene technische Kompetenzen}
\begin{itemize}
    \item Tiefes Verständnis für freeRTOS und Context-Switching
    \item Praktische Erfahrung mit Buffer-Management in Embedded Systems
    \item Architektur-Design für asynchrone Systeme
    \item Professionelle Code-Organisation in großen Projekten
    \item Debugging-Methoden für Multi-Threading-Umgebungen
\end{itemize}

\nlparagraph{Methodische Kompetenzen}
\begin{itemize}
    \item Iterative Entwicklung in Embedded-Projekten
    \item Code-Review-Prozesse in professionellen Umgebungen
    \item Systematisches Testing und Debugging
    \item Dokumentation während der Entwicklung
\end{itemize}

\section{Bewertung der Praxisstelle}
\label{sec:bewertung_praxisstelle}

\subsection{Positive Aspekte}

Die Praxisstelle zeichnete sich durch folgende Stärken aus:

\begin{itemize}
    \item \textbf{Eigenverantwortung:} Hoher Grad an selbstständiger Arbeit
    \item \textbf{Relevante Projekte:} Arbeit an produktiven Systemen mit realem Impact
    \item \textbf{Betreuung:} Konstruktives Feedback durch erfahrene Entwickler
    \item \textbf{Arbeitsklima:} Offene, kollegiale Atmosphäre
    \item \textbf{Technische Ausstattung:} Professionelle Entwicklungsumgebung
\end{itemize}

\subsection{Verbesserungspotenzial}

Identifizierte Verbesserungsmöglichkeiten:

\begin{itemize}
    \item Strukturiertere initiale Einarbeitung wäre hilfreich gewesen
    \item Klarere Anforderungsdefinition zu Projektbeginn hätte Iterationen reduziert
    \item Mehr Dokumentation im bestehenden Code würde Einarbeitung beschleunigen
\end{itemize}

Diese Punkte sind als konstruktive Anregungen zu verstehen und schmälern nicht den insgesamt sehr positiven Eindruck.

\subsection{Eignung für Studierende}

Die Praxisstelle ist besonders geeignet für Studierende mit:
\begin{itemize}
    \item Starkem Interesse an Embedded Systems
    \item Soliden Grundkenntnissen in C-Programmierung
    \item Affinität zu Hardware-naher Programmierung
    \item Bereitschaft zu selbstständigem Arbeiten
\end{itemize}

Die Praxisphase kann uneingeschränkt empfohlen werden für Studierende, die praktische Erfahrung in professioneller Embedded-Entwicklung sammeln möchten.
